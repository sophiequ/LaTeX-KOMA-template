% \documentclass[12pt,paper=a4,parskip=half,DIV=calc,oneside,%%
% headinclude,footinclude=false,open=right,bibliography=totoc]{scrartcl}
% \makeatletter
% \newcommand{\THS@linespread}{1.0}  \newcommand{\THScolorlinks}{true}
% \usepackage[utf8]{inputenc}\usepackage[ngerman,american]{babel}\usepackage{scrpage2}
% \usepackage{ifthen}\usepackage{eurosym}\usepackage{xspace}\usepackage[usenames,dvipsnames]{xcolor}
% \usepackage[protrusion=true,factor=900]{microtype}
% \usepackage{enumitem}
% \usepackage[pdftex]{graphicx}
% \usepackage{todonotes}
% \usepackage{dingbat,bbding} %% special characters
% \definecolor{DispositionColor}{RGB}{30,103,182}

% \usepackage[backend=biber,style=authoryear,dashed=false,natbib=true,hyperref=true%%
% ]{biblatex}

% \addbibresource{references-biblatex.bib} %% remove, if using BibTeX instead of biblatex

% %% overriding userdata %%
% \newcommand{\THSinsertauthor}{Karl Voit}
% \newcommand{\THSinserttitle}{LaTeX Template Documentation}
% \newcommand{\THSinsertsubject}{A Comprehensive Guide to Use the
% Template from https://github.com/tquaritsch/LaTeX-KOMA-template}
% \newcommand{\THSinsertkeywords}{LaTeX, pdflatex, template, documentation, biber, biblatex}

% \newcommand{\myLaT}{\LaTeX{}@TUG\xspace}

% %% for future use?
% % \usepackage{filecontents}
% % \begin{filecontents}{filename.example}
% % 
% % \end{filecontents}


% %% using existing TeX files %%
% \input{template/mycommands}
% \input{template/typographic_settings}
% \input{template/pdf_settings}
% \makeatother
% \begin{document}
% %% title page %%
% \title{\THSinserttitle}\subtitle{\THSinsertsubject}
% \author{\THSinsertauthor}
% \date{\today}

% \maketitle\newpage

% \tableofcontents\newpage

\documentclass[
	baseclass=scrartcl,
	biber,
	titlepage=plain_maketitle,
	addcolophon=false,
	addabstract=false,
	adddeclaration=false,
	addlistoffigures=false,
	laterality=oneside
]{tugthesis}

\THSauthor{Karl Voit}
\THStitle{LaTeX Template Documentation}
\THSsubject{{A Comprehensive Guide to Use the Template from https://github.com/tquaritsch/LaTeX-KOMA-template}}
\THSkeywords{LaTeX, pdflatex, template, documentation, biber, biblatex}

\usepackage{dingbat,bbding}
\usepackage{eurosym}
\newcommand{\myimportant}{%% mark important chapters
  \marginpar{\vspace{-1em}\rightpointleft}
}
\newcommand{\myinteresting}{\marginpar{\vspace{-2em}\PencilLeftDown}}

\begin{document}

%%---------------------------------------%%


\section{How to use this \LaTeX{} document template}

This \LaTeX{} document template from
\THSmyLaT\footnote{\url{http://LaTeX.TUGraz.at}} is based on \myacro{KOMA}
script\footnote{\url{http://komascript.de/}}. You don't need any
special \myacro{KOMA} knowledge (but it won't hurt either). It provides an easy to use and
easy to modify template. All settings are documented and many references to
additional information sources are given.


In general, there should not be any reason to modify a file in
the \texttt{template} folder. \emph{All important settings are
accessible in the main folder, mostly in the \texttt{main.tex}
file.} This way, it is easy to get what you need and you can update
the template independent of the content of the document.




The \emph{absolute minimum you should read} is listed below and 
marked with the hand symbol:\myimportant
\begin{itemize}
\item Section~\ref{sec:modifytemplate}: basic configuration of this template.
\item Section~\ref{sec:howtocompile}: how to generate the \myacro{PDF} file
\item Section~\ref{sec:references}: using biblatex (instead of bibtex)
\end{itemize}

In order to get a perfect resulting document and to get an
exciting experience with this template, you should definitely consider reading
following sections which are also marked with the pencil symbol:\myinteresting
\begin{itemize}
\item Section~\ref{sec:extending-template}: extend the template with
  your own usepackages, newcommands, and so forth
\item Section~\ref{sec:mycommands}: pre-defined commands to make your life easier (e.g., including graphics)
\item Section~\ref{sec:myacro}: how to do acronyms (like \myacro{ACME}) beautifully
\item Section~\ref{sub:csquotes}: how to \enquote{quote} text and use parentheses correctly
\end{itemize}

The other sections describe all other settings for the sake of completeness. This is
interesting for learning more about \LaTeX{} and modifying this template to a higher level of detail.


\newpage
\subsection{Six Steps to Customize Your Document}\myimportant
\label{sec:modifytemplate}

This template is optimized to get to the first draft of your thesis
very quickly. Follow these instructions and you get most of your
customizing done in a few minutes:

\newcommand{\myfile}[1]{\texttt{\href{file:#1}{#1}}}

\begin{enumerate}
\item Modify settings in \texttt{main.tex} to meet your requirements:
  \begin{itemize}
  \item Basic settings
    \begin{itemize}
    \item Paper size, languages, font size, citation style,
          title page, and so forth
    \end{itemize}
  \item Document metadata
    \begin{itemize}
    \item Preferences like \verb+myauthor+, \verb+mytitle+, and so forth
    \end{itemize}
  \end{itemize}
\item Replace \myfile{figures/institution.pdf} with the logo of
your institution in either \myacro{PDF} or \myacro{PNG}
format.\footnote{Avoid \myacro{JPEG} format for
computer-generated (pixcel-oriented) graphics like logos or
screenshots in general. The \myacro{JEPG} format is for
photographs \emph{only}.}
\item Further down in \myfile{main.tex}:
  \begin{itemize}
  \item Create your desired structure for the chapters
        (\verb+\include{introduction}+, \verb+\include{evaluation}+, \ldots)
  \end{itemize}
\item Create the \TeX{} files and fill your content into these files you defined in the previous step.
\item Optionally: Modify \myfile{colophon.tex} to meet your situation.
  \begin{itemize}
  \item Please spend a couple of minutes and think about putting your work
        under an open license\footnote{\url{https://creativecommons.org/licenses/}}
        in order to follow the spirit of Open Science\footnote{\url{https://en.wikipedia.org/wiki/Open_science}}.
  \end{itemize}
\item In case you are using \myacro{GNU} make\footnote{If you
      don't know, what \myacro{GNU} make is, you are not using it (yet).}: 
      Put your desired \myacro{PDF} file name in the second line of file
   \myfile{Makefile}
   \begin{itemize}
   \item replace \enquote{Projectname} with your filename
   \item do not use any file extension like \texttt{.tex} or \texttt{.pdf}
   \end{itemize}
\end{enumerate}




\subsection{License}
\label{sec:license}

This template is licensed under a Creative Commons Attribution-ShareAlike 3.0 Unported (CC BY-SA 3.0)
        license\footnote{\url{https://creativecommons.org/licenses/by-sa/3.0/}}:
    \begin{itemize}
    \item You can share (to copy, distribute and transmit) this template.
    \item You can remix (adapt) this template.
    \item You can make commercial use of the template.
    \item In case you modify this template and share the derived
          template: You must attribute the template such that you do not
          remove \mbox{(co-)}authorship of Karl Voit and you must not remove
          the URL to the original repository on 
          github\footnote{\url{https://github.com/novoid/LaTeX-KOMA-template}}.
    \item If you alter, transform, or build a new template upon 
          this template, you may distribute the resulting 
          template only under the same or similar license to this one. 
    \item There are \emph{no restrictions} of any kind, however, related to the
          resulting (PDF) document!
    \item You may remove the colophon (but it's not recommended).
    \end{itemize}




\subsection{How to compile this document}\myimportant
\label{sec:howtocompile}

I assume that compiling \LaTeX{} documents within your software
environment is something you have already learned. This template is
almost like any other \LaTeX{} document except it uses
state-of-the-art tools for generating things like the list of
references using biblatex/biber (see
Section~\ref{sec:references} for details). Unfortunately, some \LaTeX{} editors
do not support this much better way of working with bibliography
references yet. This section describes how to compile this template.

\subsubsection{Compiling Using a \LaTeX{} Editor}

Please do select \myfile{main.tex} as the \enquote{main project file} or make
sure to compile/run only \myfile{main.tex} (and not \myfile{introduction.tex}
or other \TeX{} files of this template).

Choose \texttt{biber} for generating the references. Modern LaTeX{}
environments offer this option. Older tools might not be that up-to-date
yet.


\subsubsection{Activating \texttt{biber} in the \LaTeX{} editor TeXworks}
\label{sec:biberTeXworks}

The \href{https://www.tug.org/texworks/}{TeXworks} editor is a very
basic (but fine) \LaTeX{} editor to start with. It is included in
\href{http://miktex.org/}{MiKTeX} and
\href{http://miktex.org/portable}{MiKTeX portable} and supports
\href{https://en.wikipedia.org/wiki/Syntax_highlighting}{syntax
  highlighting} and
\href{http://itexmac.sourceforge.net/SyncTeX.html}{SyncTeX} to
synchronize \myacro{PDF} output and \LaTeX{} source code.

Unfortunately, TeXworks shipped with MiKTeX does not support compiling
using \texttt{biber} (biblatex) out of the box. Here is a solution to
this issue. Go to TeXworks: \texttt{Edit} $\rightarrow$
\texttt{Preferences~\ldots} $\rightarrow$ \texttt{Typesetting} $\rightarrow$
\texttt{Processing tools} and add a new entry (using the plus icon):

\begin{tabbing}
  Arguments: \= foobar  \kill
  Name:      \> \verb#pdflatex+biber# \\
  Program:   \> \emph{find the \texttt{template/pdflatex+biber.bat} file from your disk} \\
  Arguments: \> \verb+$fullname+ \\
             \> \verb+$basename+
\end{tabbing}

Activate the \enquote{View PDF after running} option.

Close the preferences dialog and you will now have an additional
choice in the drop down list for compiling your document. Choose the
new entry called \verb#pdflatex+biber# and start a happier life with
\texttt{biber}.

In case your TeXworks has a German user interface, here the key
aspects in German as well:

\begin{otherlanguage}{ngerman}

  \texttt{Bearbeiten} $\rightarrow$ \texttt{Einstellungen~\ldots} $\rightarrow$
  \texttt{Textsatz} $\rightarrow$ \texttt{Verarbeitungsprogramme} $\rightarrow$
  + \emph{(neues Verarbeitungsprogramm)}:

\begin{tabbing}
  Befehl/Datei: \= foobar  \kill
    Name: \> pdflatex+biber \\
    Befehl/Datei: \> \emph{die \texttt{template/pdflatex+biber.bat} im Laufwerk suchen} \\
    Argumente: \> \verb+$fullname+ \\
               \> \verb+$basename+
\end{tabbing}

\enquote{PDF nach Beendigung anzeigen} aktivieren.

\end{otherlanguage}


\subsubsection{Compiling Using \myacro{GNU} make}

With \myacro{GNU}
make\footnote{\url{https://secure.wikimedia.org/wikipedia/en/wiki/Make\_\%28software\%29}}
it is just simple as that: \texttt{make pdf}

Several other targets are available. You can check them out by
executing: \texttt{make help}


\subsubsection{Compiling in a Text-Shell}

To generate a document using \texttt{Biber}, you can stick to
following example:
\begin{verbatim}
pdflatex main.tex
biber main
pdflatex main.tex
pdflatex main.tex
\end{verbatim}




\subsection{How to get rid of the template documentation}

Simply remove the files \verb#Template_Documentation.pdf# and 
(if it exists)  \verb#Template_Documentation.tex# in the main folder 
of this template.

\subsection{What about modifying or extending the template?}\myinteresting
\label{sec:extending-template}

This template provides an easy to start \LaTeX{} document template with sound
default settings. You can modify each setting any time. It is recommended that
you are familiar with the documentation of the command whose settings you want
to modify.

It is recommended that for \emph{adding} things to the preamble (newcommands,
setting variables, defining headers, \dots) you should use the file
\texttt{main.tex}. 
There are comment lines which help you find the right spot.
This way you still have the chance to update your \texttt{template}
folder from the template repository without losing your own added things.

The following sections describe the settings and commands of this template and
give a short overview of its features.

\subsection{How to change the title page}

This template comes with a variety of title pages. They are located in
the folder \texttt{template}. You can switch to a specific title
page by including the corresponding title page file in the file
\texttt{main.tex}.

Please note that you may not need to modify any title page document by
yourself since all relevant information is defined in the file
\texttt{main.tex}.
